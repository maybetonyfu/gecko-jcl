\documentclass{article}


\newcommand{\pointRaised}[2]{\medskip \noindent 
               \paragraph{#1} \textsl{#2}} 
\newcommand{\reply}{\noindent \textbf{Response}:\ } 

\begin{document}

Dear Editor,

Thank you and the reviewers for the opportunity to revise our manuscript “GeckoGraph: A Visual Language for Polymorphic Types.”
We acknowledge the constructive and insightful feedback provided. We think they help improve the clarity, structure, and balance of our paper.
In this revised version, we have addressed all reviewer comments and made corresponding improvements throughout the manuscript. Below is a summary of major changes, followed by detailed point-by-point responses to each reviewer.

We hope our revisions satisfactorily address all concerns and that the paper is now suitable for publication in the Journal of Computer Languages.

Sincerely,

Authors of "GeckoGraph: A Visual Language for Polymorphic Types"
\section{Summary of Major Revisions}

\paragraph{Balanced claims} We have moderated the tone in the \textit{Abstract}, \textit{Introduction}, and \textit{Conclusion} to present a more balanced and evidence-based discussion, explicitly acknowledging both the strengths and limitations of our findings.

\paragraph{Terminology refinement} We have replaced references to the user study as a “game” with more accurate terms such as “gamified user study” or simply “user study,” to avoid overstating the level of interactivity.

\paragraph{New background section} We have added a new section, \textit{Type Systems in Functional Languages}, following the \textit{Introduction} to provide theoretical background and better motivate the design of GeckoGraph.

\paragraph{Section structure overviews} We have introduced short structural overviews at the beginning of the \textit{Related Work} and \textit{Discussion} sections to help guide readers through their organization.

\paragraph{Expanded related work} We have extended the \textit{Related Work} section to include additional references, notably the paper “Just TYPEical: Visualizing Common Function Type Signatures in R,” and used it to better position GeckoGraph among related visualization approaches.

\paragraph{Enhanced discussion} The \textit{Discussion} section has been revised to include more concrete examples demonstrating potential applications of GeckoGraph.

\paragraph{Concise figure captions} All figure captions have been shortened for clarity, readability, and stylistic consistency.

\paragraph{Improved flow between sections} A transitional paragraph has been added to connect the \textit{Design Goals} and \textit{Detailed GeckoGraph Features} subsections in Section~3.1, improving the logical flow of ideas.

\paragraph{Clarified explanation of type classes} The explanation of type classes in Section~3.1 has been rewritten in simpler language with improved examples. Figure~3 now includes concrete illustrations to support the description.

\paragraph{Removed unused figure} We removed the previous Figure~4, as it did not contribute additional information to the paper.

\paragraph{Cosmetic and consistency edits} Grammar, spelling, LaTeX syntax, and stylistic consistency have been carefully reviewed and corrected throughout the manuscript.

\section{Detailed Responses to Reviewers}

\subsection*{Response to Reviewer 1}

\pointRaised{Comment 1:}{The manuscript’s tone in the abstract and conclusion appears overly strong given the presented evidence.}

\reply We appreciate this observation. The language in the \textit{Abstract}, \textit{Introduction}, and \textit{Conclusion} has been moderated to ensure that claims are proportionate to the results and that both positive and negative aspects of our findings are clearly presented.

\pointRaised{Comment 2:}{The term “game” used to describe the user study could be misleading.}

\reply We agree. The tasks in our study do not fully meet the criteria for a “game.” We now describe it more accurately as a “gamified user study” or simply a “user study,” emphasizing its educational and experimental nature.

\pointRaised{Comment 3:}{Please clarify the explanation of type classes.}

\reply The subsection on type classes (Section~3.1) has been rewritten for clarity and supported by clearer examples. Figure~3 has also been improved to include illustrative examples that reinforce our description of GeckoGraph.

\pointRaised{Comment 4:}{Issues with figures and references.}

\reply The previous Figure~4 has been removed. Incorrect cross-references such as “Fig.~2.2” have been corrected (now “Fig.~8”). Repetitions such as “Appendix Appendix~A” have been fixed.

\pointRaised{Comment 5:}{Temper the narrative of qualitative results—the data show neutral user opinions.}

\reply We have clarified the neutrality of participants’ overall feedback and added clearer indicators of sentiment (e.g., “slightly positive,” “moderately negative”) to reflect the scale values more accurately.

\pointRaised{Comment 6:}{Clarify speculative claims in the Discussion.}

\reply We now explicitly mark speculative interpretations in Section~6 (\textit{Discussion}) with leading phrases such as “We hypothesize that...”.

\pointRaised{Comment 7:}{Wording and grammar issues.}

\reply We have corrected numerous stylistic and grammatical errors, including:
“expressiveness power” → “expressiveness”,
“high-order function” → “higher-order function”,
and “GechoGraph” → “GeckoGraph.”  
Unclear terms such as “tangibly” and “regular higher-order function” have been clarified or replaced with precise expressions and supported with examples.  
We also removed unnecessary spaces before footnote labels.

\pointRaised{Comment 8:}{Clairify the syntactic sugar of curried function (Section 3)}

\reply We explicitly clarify that function notation \texttt{a -> b -> c } desugars to \texttt{a -> (b -> c) }, which has a consistent representation in GeckoGraph.

\subsection*{Response to Reviewer 2}

\pointRaised{Comment 1:}{The paper lacks sufficient background on type systems in functional languages.}

\reply We have added a new section titled \textit{Type Systems in Functional Languages} immediately after the \textit{Introduction}. This section introduces key theoretical concepts and motivates the design rationale for GeckoGraph.

\pointRaised{Comment 2:}{The related work section could benefit from additional references and a clearer structure.}

\reply We have expanded and reorganized the \textit{Related Work} section, including new references such as “Just TYPEical: Visualizing Common Function Type Signatures in R.” A short overview paragraph now outlines the section’s structure and focus.

\pointRaised{Comment 3:}{The discussion section is abstract and lacks concrete examples.}

\reply We have revised the \textit{Discussion} section to include additional examples illustrating practical applications of GeckoGraph. Figure~14 has been added to support this discussion.

\pointRaised{Comment 4:}{Begin Sec.~5 (“Discussion”) and Sec.~6 (“Related Work”) with overview paragraphs explaining purpose and structure.}

\reply Overview paragraphs have been added at the start of both sections to help orient readers and clarify how each section is organized.

\pointRaised{Comment 5:}{Usage of figures.Some figures (1,3,10,14 previous version) captions are too long. Figure 4 is hard to understand and its purpose is unclear.}

\reply Long figure captions have been shortened for conciseness. The previous Figure~4, which did not contribute new insights, has been removed.

\pointRaised{Comment 6:}{The paper’s flow between sections feels abrupt.}

\reply We have added a transitional paragraph between the \textit{Design Goals} and \textit{Detailed GeckoGraph Features} subsections in Section~3 to improve the narrative flow.

\pointRaised{Comment 7:}{Minor issues with grammar, formatting, and coherence.}

\reply The manuscript has been carefully proofread to correct grammatical and LaTeX syntax errors and improve readability.  
Examples include correcting figure references (e.g., “Fig.~2.2” → “Fig.~8”), fixing duplicate words (“Appendix Appendix~A” → “Appendix~A”), and removing extra spaces before footnote labels.

\pointRaised{Comment 8:}{Clarify user tasks and results in Figures 9 and 10.}

\reply We have clarified that the user in Figures~9 and~10 did not reach the correct solution and explained how GeckoGraph’s visual feedback can help identify and correct such mistakes. We also fixed a typo in Figure~9, changing “runHero z” to “runZero z.”

\pointRaised{Comment 9:}{Clarify whether participants indicate their experience once or before each run.}

\reply We clarified that participants report their programming experience once, before beginning the tasks.

\pointRaised{Comment 10:}{Distinguish between empirical findings and authors’ interpretations in the Discussion.}

\reply We now explicitly distinguish between observed results and speculative interpretations in the \textit{Discussion}, using clear phrases such as “We hypothesize that...” to signal interpretive commentary.

\pointRaised{Comment 11:}{Add reference for “between-subject design.”}

\reply We now correctly cite Fisher, R.A. {\it The Design of Experiements} for the phase "between-subject design."


\pointRaised{Comment 12:}{Wording and grammar issues.}

\reply We have corrected numerous stylistic and grammatical errors, including:
Unclear terms such as “approximity” and “during type errors” have been clarified or replaced with precise expressions and supported with examples.  


\end{document}