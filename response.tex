\documentclass{article}


\newcommand{\pointRaised}[2]{\medskip \noindent 
               \paragraph{#1} \textsl{#2}} 
\newcommand{\reply}{\noindent \textbf{Response}:\ } 

\begin{document}

Dear Editor,

Thank you and the reviewers for the opportunity to revise our manuscript “GeckoGraph: A Visual Language for Polymorphic Types.”
We acknowledge the constructive and insightful feedback provided. We think they help improve the clarity, structure, and balance of our paper.
In this revised version, we have addressed all reviewer comments and made corresponding improvements throughout the manuscript. Below is a summary of major changes, followed by detailed point-by-point responses to each reviewer.

We hope our revisions satisfactorily address all concerns and that the paper is now suitable for publication in the Journal of Computer Languages.

Sincerely,

Authors of "GeckoGraph: A Visual Language for Polymorphic Types"

\section{Summary of Major Revisions}

\paragraph{Balanced claims} – We have now moderated the tone in the Abstract, Introduction, and Conclusion to be less ambitious and more balanced, explicitly acknowledging both the strengths and limitations of our findings.

\paragraph{Terminology refinement} – We have now replaced references to the user study as a “game” with more precise wording, such as “gamified user study” or simply “user study.”

\paragraph{New background section} – We have now added a new section, Type Systems in Functional Languages, after the Introduction to provide theoretical background and motivate the design of GeckoGraph.

\paragraph{Section structure overviews} – We have now added section structure overviews at the beginning of the Related Work and Discussion sections to better guide the reader.

\paragraph{Expanded related work} – We have now discussed the relevant study “Just TYPEical: Visualizing Common Function Type Signatures in R” to strengthen the discussion of related approaches and compare to our own contributions.

\paragraph{Enhanced discussion} – We have now improved the Discussion section with more concrete examples demonstrating the potential applications of GeckoGraph.

\paragraph{Concise figure captions} – We have shortened our figure captions throughout the manuscript for readability and stylistic consistency.

\paragraph{Improved flow between sections} – We have added a transitional paragraph linking the Design Goals and Detailed GeckoGraph Features subsections in Section 3.1.

\paragraph{Clarified explanation of type classes} – We have now refined the text in Section 3.1 to improve clarity when explaining type classes.

\paragraph{Removed unused figure} – We removed Figure 4 from the previous version, as it did not add new information to the paper.

\paragraph{Cosmetic and consistency edits} – We improved our grammar, spelling, LaTeX syntax, and overall logical coherence across the manuscript.

\section{Detailed Responses to Reviewers
Response to Reviewer 1}

\pointRaised{Comment 1:}{ The manuscript’s tone in the abstract and conclusion appears overly strong given the presented evidence.}

\reply We appreciate this observation. We have moderated the language in the Abstract, Introduction, and Conclusion to ensure our claims are proportionate to the results, and to present both positive and negative aspects of our findings.

\pointRaised{Comment 2:}{The term “game” used to describe the user study could be misleading.}

\reply We agree that we don't meet all the expectations for a game, and it's more appropriate to frame it as "an entertaining coding challenge". We changed our word of choice for “game” to “gamified user study” or simply, “user study”.  

\pointRaised{Comment 3:}{The paper lacks sufficient background on type systems in functional languages.}

\reply We have added a new section titled Type Systems in Functional Languages immediately following the Introduction. This section introduces relevant concepts and better motivates the design of GeckoGraph.

\pointRaised{Comment 4:}{Please clarify the explanation of type classes.}

\reply The subsection explaining type classes (Section 3.1) has been rewritten for clarity, with simpler language and improved examples.

\section{Response to Reviewer 2}

\pointRaised{Comment 1:}{The related work section could benefit from additional references and a clearer structure.}

\reply We have expanded the Related Work section to include “Just TYPEical: Visualizing Common Function Type Signatures in R” and added an introductory overview to outline its structure.

\pointRaised{Comment 2:}{The discussion section is abstract and lacks concrete examples.}

\reply We have revised the Discussion section to include several new examples illustrating potential applications of GeckoGraph.

\pointRaised{Comment 3:}{Some figures and captions are overly detailed.}

\reply We have shortened all figure captions for conciseness and removed Figure 4 from the earlier version, as it did not contribute additional insight.

\pointRaised{omment 4:}{The paper’s flow between sections feels abrupt.}

\reply We added a transitional paragraph between the Design Goals and Detailed GeckoGraph Features sections to improve narrative flow.

\pointRaised{Comment 5:}{Minor issues with grammar, formatting, and coherence.}

\reply We have now carefully proofread the entire manuscript, correcting grammatical errors, LaTeX syntax issues, and improving overall readability.




\end{document}

